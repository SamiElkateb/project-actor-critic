\section{Introduction}

\paragraph{}
Acteur-critique est une méthode d'apprentissage par renforcement de type policy-gradient.
Contrairement à d'autres méthodes de type policy-gradient, acteur-critique utilise la 
différence temporelle ainsi qu'une approximation de la fonction de valeur d'état $\hat{v}(s_t)$
pour apprendre la politique. Dans ce document,
nous détaillerons le fonctionnement de la méthode acteur-critique,
ses avantages et ses inconvénients. Puis,
nous examinerons une implémentation spécifique pour le jeu de Pong et explorerons des pistes d'amélioration.

