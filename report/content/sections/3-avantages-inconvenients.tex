\section{Avantages \& Inconvénients}

\subsection{Avantages}

\paragraph{}
La méthode Acteur-Critique combine les avantages des méthodes de type value-based (Q-Learning, Sarsa) et policy-based:
\begin{itemize}
  \item Elle offre une meilleure convergence que les méthodes basées uniquement sur la politique, grâce à l'utilisation de la différence temporelle.
  \item Elle ne nécessite pas de données labellisés contrairement à l'approximation de la fonction de valeur par apprentissage supervisé.
\end{itemize}
\subsection{Inconvénients}
\paragraph{}
Elle présente cependant quelques inconvénients:
\begin{itemize}
  \item Elle nécessite d'apprendre à la fois une fonction de valeur et une politique, ce qui augmente la complexité et le coût de calcul de l'algorithme.
  \item Des erreurs dans l'estimation de la fonction de valeur par le critique peuvent entraîner de mauvaises mises à jour de la politique par l'acteur.
\end{itemize}

