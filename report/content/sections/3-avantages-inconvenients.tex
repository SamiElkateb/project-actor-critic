\section{Avantages \& Inconvénients}

\subsection{Avantages}

\paragraph{}
\begin{samepage}
La méthode Acteur-Critique combine les avantages des méthodes de type value-based (Q-Learning, Sarsa) et policy-based:
\end{samepage}
\begin{itemize}
  \item Elle offre une meilleure convergence que les méthodes basées uniquement sur la politique, grâce à l'utilisation de la différence temporelle.
  \item Ne nécessite pas de données labellisés contrairement à l'approximation de la fonction de valeur par apprentissage supervisé.
  \item Contrairement à Q-Learning ou SARSA, elle peut fonctionner dans un espace d'état très large ou continu.
\end{itemize}
\subsection{Inconvénients}
\paragraph{}
Elle présente cependant quelques inconvénients~:
\begin{itemize}
  \item Nécessite moins d'épisodes pour converger, mais les épisodes sont plus longs.
  \item Nécessite d'apprendre à la fois une fonction de valeur et une politique, ce qui augmente la complexité et le temps de calcul de l'algorithme.
  \item Moins efficace que Q-Learning ou SARSA dans des espaces restreints et discret.
\end{itemize}

